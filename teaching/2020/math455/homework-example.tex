\documentclass{amsart}

\usepackage{amsmath}
\usepackage{amsthm}
\usepackage{amssymb}
\usepackage{mathtools}

%% for the exercise and solution environements
\theoremstyle{plain}
\ifdefined\exercise \else \newtheorem{exercise}{Exercise} \fi
\newenvironment{solution}{\begin{proof}[Solution]}{\end{proof}}


\newcommand\Nbb{\mathbb{N}}
\newcommand\Rbb{\mathbb{R}}
\newcommand\powerset{\mathcal{P}}


\title{Math455: Homework 0}
\author{Kameryn J Williams}

\date{January 10, 2019}

\begin{document}

\maketitle

\begin{exercise}
Show that there is no bijection between $\Nbb$ and $\Rbb$.
\end{exercise}

\begin{solution}
It was proven in class on Monday, January 6 that $\Rbb$ is in bijective correspondence with $\powerset(\Nbb)$. Accordingly, it suffices to prove that there is no bijection between $\Nbb$ and $\powerset(\Nbb)$. To this end, consider an arbitrary function $f \colon \Nbb \to \powerset(\Nbb)$. Define $D \subseteq \Nbb$ as $D = \{ n \in \Nbb : n \not \in f(n) \}$. I claim that $D$ is not in the range of $f$. To see this, suppose it were the case that $D = f(n)$ for some $n$. If $n \in D$, then by the definition of $D$ we would have that $n \not \in f(n) = D$, a contradiction. On the other hand, if $n \not \in D = f(n)$, then again by by the definition of $D$ we would have that $n \in D$. Either way, we get a contradiction. So it must be that our arbitrary $f$ is not surjective, and hence fails to be bijective.
\end{solution}

\end{document}

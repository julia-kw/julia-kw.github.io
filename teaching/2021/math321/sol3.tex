\documentclass[10pt]{amsart}
\usepackage[margin=1in]{geometry}
\usepackage{amsmath}
\usepackage{amsthm}
\usepackage{amssymb}
\usepackage{mathtools}
\usepackage{graphicx}
\usepackage[multiple]{footmisc}
\usepackage{multicol}
%\usepackage{enumerate}
\usepackage{enumitem}

%% make the problem environment look how I want
\theoremstyle{definition}
\newtheorem{problem}{Problem}

\title{Math 321: Homework 3 Solution}

\begin{document}

\maketitle

% get the numbering right since I'm only showing Problem 4
\setcounter{problem}{3}
\begin{problem}[Exercise 3.12 from the textbook]
Prove that a positive integer is square-free if and only if all the exponents in its prime factorization are $1$.
\end{problem}

\begin{proof}[Solution]
I prove both directions by contrapositive.

$(\Rightarrow)$ Consider a positive integer $n$ and suppose that there is a prime $p$ in its prime factorization which has an exponent $m>1$. Then, $n$ is a multiple of $p^m$ which in turn is a multiple of $p^2$. So $n$ is not square-free.

$(\Leftarrow)$ Suppose that $n$ is not square-free. That is, $n = a^2b$ for some integers $a > 1$ and $b$. Suppose $p$ is in the prime factorization of $a$, with some exponent $m$. And let $\ell$ be the largest integer so that $p^\ell$ divides $b$. (Possibly $\ell = 0$, which happens when $p$ does not divide $b$.) Then, $p^{2m+\ell}$ appears in the prime factorization of $n$. So $n$ has a prime in its factorization whose exponent is not $1$.
\end{proof}


\end{document}

